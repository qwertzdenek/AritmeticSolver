\documentclass[a4paper, 12pt]{article}

\usepackage[czech]{babel}
\usepackage[utf8]{inputenc}
\usepackage{lmodern} % fonty Latin Modern
\usepackage{cmap}    % podpora kopírování v PDF
\usepackage[T1]{fontenc} % Nastavení kódování fontu na formát T1.

\usepackage{graphicx}
\usepackage{listings}
\usepackage{url}

\DeclareUrlCommand\url{\def\UrlLeft{<}\def\UrlRight{>} \urlstyle{tt}}

\lstset{language=C}

\begin{document}
\begin{titlepage}
\includegraphics{fav_cmyk.pdf}
\vfill
\begin{center}
{\huge Formální jazyky a překladače}\\[3ex]
{\Large Interpret jazyka LISP}
\end{center}
\vfill
\begin{tabbing}
Vypracoval: \hspace{1ex}\=Zdeněk Janeček\kill
Vypracoval: \>Zdeněk \textsc{Janeček}\\[1ex]
Datum:\> \today
\end{tabbing}
\end{titlepage}

\tableofcontents

\section{Zadání}
Naprogramujte v~jazyce C přenositelnou \emph{konzolovou aplikaci}, která
bude fungovat jako jednoduchý interpret podmnožiny jazyka LISP.
Vstupem budou příkazy v~jazyce LISP. Výstupem pak výsledek vyhodnocení
každého výrazu.

Program se bude spouštět příkazem \textsf{lisp [soubor]}. Symbol
\textsf{[soubor]} zastupuje nepovinný parametr -- název vstupního
souboru se seznamem výrazů v~jazyce LISP. Není-li první parametr
uveden, program bude fungovat v~interaktivním módu, kdy se příkazy
budou provádět přímo zadáním z~konzole do programu.

Úkolem Vámi vyvinutého programu tedy je:
\begin{enumerate}
\item Při spuštění bez parametru bude čekat na vstup od uživatele.
  Zadaný výraz vyhodnotí a bude vyžadovat další vstup, dokud nebude
  uveden výraz \texttt{(quit)}.
\item Při spuštění s~parametrem načte zadaný vstupní soubor, každý
  výraz v~něm uvedený vypíše na obrazovku, okamžitě vyhodnotí a
  výsledek vypíše na obrazovku. Po zpracování posledního výrazu dojde
  k~ukončení programu. Proto nemusí být jako poslední výraz uveden
  výraz \texttt{(quit)}. Na jedné řádce v~souboru může být uvedeno
  více samostatných výrazů a program je musí být schopen správně
  zpracovat.
\end{enumerate}

Program může být během testování spuštěn například takto:
\begin{verbatim}
  $ ./lisp test.lisp # přímé čtení souboru
  $ cat test.lisp | ./lisp # stdin roura
  $ ./lisp # interaktivně
\end{verbatim}

\section{Problematika}
Jazyk LISP je znám svojí jednoduchou gramatikou.
Jeho základ lze popsat třemi základními neterminály a počátečním
symbolem START takto: 

\medskip

\begin{tabular}{r@{$\quad\longrightarrow\quad$}l}
START & LIST START | LIST\\
LIST & (LIST\_IN)\\
LIST\_IN & fce ARG\\
ARG & LIST ARG | term ARG | e
\end{tabular}

Oddělil jsem terminální symboly na názvy funkcí a termy obsahující
zbylá čísla a proměnné. Generovaný jazyk je například:

\begin{verbatim}
(fce LIST ARG) -> (fce (LIST_IN) ARG)
-> (fce (fce ARG) ARG) -> (fce (fce term))
\end{verbatim}

Nenastane tedy případ, který by skončil se zakázanou derivací
se seznamem bez funkce.
Jednoznačně totiž platí, že za každou otevírací závorkou musí
vzniknout funkce (+, -, quote, set, if...), ale čísla a proměnné
jsem označil terminálním symbolem term. Ta vznikají jen řetězením
symbolu ARG, který následuje po funkci.

\subsection{Lexikální analýza}
Při návrhu programu jsem se řídil gramatikou, kterou jsem od
základního návrhu rozšířil o~podporu výpisu bez vyhodnocování (quote).
Její podoba je na obrázku \ref{fig:gram}.

\begin{figure}
\centering
\begin{tabular}{r@{$\quad\longrightarrow\quad$}l}
START & LIST | 'ARG\_SYM\\
LIST & (LIST\_IN)\\
LIST\_IN & fce ARG | e\\
ARG & ARG\_SYM ARG | 'ARG\_SYM ARG | e\\
ARG\_SYM & LIST | var | num
\end{tabular}
\caption{Gramatika použitého jazyka}
\label{fig:gram}
\end{figure}

Oddělil jsem také neterminální symbol argumentu pro jeho opakované
použití. Dovolené je hned jako počáteční symbol zavolat quote,
ale jen v~jeho zkrácené formě. Zbytek zůstává stejný. Oba symboly
fce a var lze definovat pro nové symboly. Pro lexikální analýzu
je důležité umět rozlišit funkci od proměnné. Lexikální analyzátor
rozlišuje jen vnitřní funkce a ostatní symboly jsou jen proměnné.

Detekuji symboly definované v~\texttt{lexa.h} (tabulka
\ref{tab:atom}).

\begin{table}
\centering
\begin{tabular}{|r|l|}
\hline
Název & Popis\\ \hline\hline
AT\_FCE & funkce\\ \hline
AT\_VAR & proměnná\\ \hline
AT\_NUM & číslo\\ \hline
AT\_LBRACKET & otevírací závorka\\ \hline
AT\_RBRACKET & zavírací závorka\\ \hline
AT\_QUOTE & výpis bez vyhodnocení\\ \hline
AT\_UNKNOWN & neznámý symbol\\
\hline
\end{tabular}
\caption{Rozeznávané atomy}
\label{tab:atom}
\end{table}

Zpracování vstupu probíhá sekvenčně pomocí \verb+lexa_next(atom *sym)+.
V~jednu chvíli mám uložen právě jeden symbol. V~tomto ohledu se nijak
neliší načítání ze standardního vstupu a ze souboru. Jeden atom
je uložen ve struktuře:

\begin{lstlisting}
typedef struct
{
    int type;
    union
    {
        char string[28];
        int value;
    };
} atom;
\end{lstlisting}

Před začátkem čtení je třeba nejprve lexikální analyzátor inicializovat
funkcí \verb+lexa_state lexa_init(lexa_state *s)+, která nastaví vnitřní
stav analyzátoru na předaný stav. Stav klasifikátoru jsem zavedl kvůli
podpoře rekurze. Obsahuje aktuální čtecí stream, poslední načtený symbol
a znak. První symbol by neměl být prázdný a je to nějaký imaginární
bílý znak.

Pro zpracování symbolů používám několik pomocných funkcí k~tvorbě vnitřní
struktury:

\begin{description}
  \item[is\_operator] testuje existenci operátoru
  \item[write\_atom\_fce] atom funkce
  \item[write\_atom\_var] atom proměnné
  \item[write\_atom\_num] atom samotného čísla
  \item[write\_atom\_quote] zkrácený quote symbol \verb+'+
\end{description}

Samotná funkce \texttt{lexa\_next} zajišťuje čtení nových symbolů.
Nejdříve se vypořádám s~prázdnými znaky a pokud nejsem na konci,
rozhodnu se podle znaku \texttt{lchar}. Platí obvyklé pravidlo
že identifikátor nesmí začínat číslem, ale je dovolené třeba podtržítko.
Nejdříve zjistím, zda se jedná o~číslo s~unárním operátorem, pak
interní funkci a jinak se musí jednat o~proměnnou. V~případě
že symbol není rozpoznán, je vrácen jako neznámý.

Funkce \texttt{is\_operator(char o)} vrací \texttt{true} když
symbol je operátor. Nyní máme vytvořený symbol \texttt{csym}, ten
stačí nakopírovat na místo, které nám bylo dodáno zvenčí.

Aktuální, symbol dostaneme voláním
\verb+void lexa_get(atom *sym)+. Tím se analyzátor nikam nepřesouvá.
Pořadí volání funkcí je následující: \texttt{init}, \texttt{next} a
libovolně volaný \texttt{get}.

\subsection{Syntaktická analýza}
Zpracování výrazu se může nacházet ve třech stavech a to OK\_CODE,
ERROR\_CODE a END\_CODE. OK\_CODE indikuje že vše proběhlo v~pořádku,
ERROR\_CODE indikuje že nastala, chyba. Při chybě je nahrán chybový
řetězec do sdíleného místa \texttt{error\_message} a okamžitě ukončeno
vykonávání aktuální funkce. Volající funkce je povinna předat chybu dále.

Každý zpracovávaný vstup začíná funkcí \texttt{start()} stejně jak
je naznačeno v~gramatice na obrázku \ref{fig:gram}.

V~případě že jedná quote výraz, použiji funkce s~prefixem
\texttt{quote\_}. Od běžných funkcí se liší tím, že výraz nevyhodnocují
a volají jen další \texttt{quote\_} podprogramy.

V~opačném případě se zavolá \texttt{list(char *res)}. Ten otestuje
zda se jedná o~závorkovaný výraz a vyhodnotí obsaženou funkci.
Její vyhodnocení probíhá uvnitř seznamu, kde na prvním místě je
funkce a pak její argumenty. Funkce může být také definovaná pomocí
\textsf{defun}.

Aritmetické výrazy byly dvojího typu. Buď aritmetické nebo
booleovské. Každý aritmetický výraz je vyhodnocen zleva doprava
postupným aplikováním operátoru na prvním místě. Musel jsem tedy na
počátku zvolit neutrální prvek k~této operaci. Spočtená hodnota se
ukládá do mezi-proměnné a na konci tam zůstane výsledek. Využil jsem
s~výhodou datový typ ukazatel na funkci.

Každý operátor má svou vlastní funkci jako tato:
\begin{lstlisting}
int addii(int a, int b)
{
  return a + b;
}
\end{lstlisting}

Podle toho jaký operátor funkce \texttt{vyraz(char *out)} najde,
přiřadí příslušný ukazatel na funkci \verb+int (*op)(int, int)+.

Číslem se rozumí triviální vstup čísla a závorkovaný výraz zavolá
rekurzivně seznam. Jednotlivé identifikátory představují
vnitřní příkazy interpretu až na poslední možnost, kdy se vyhledá
proměnná v~seznamu uložených proměnných.

Při změně hodnoty která v~seznamu již je, nepotřebuji najít a mazat
starou hodnotu. Protože přidávám vždy na konec a prohledávám odzadu,
starou hodnotu nikdy nenajdu. Do seznamu vkládám v~konstantním čase.
Klíčem položky seznamu je její název a hodnotou je struktura
\verb+member_t+ obsahující typ položky (funkce nebo proměnná)
a jejich data.

\section{Používání}
K~překladu aplikace není třeba žádné zvláštní závislosti. Potřeba je
pouze překladač jazyka C. Testován byl překladač \textsf{GCC} a \textsf{Clang}.
Přibalen je soubor \textsf{Makefile}. Po překladu pomocí programu
\textsf{make} vznikne spouštěcí soubor \textsf{lisp}.

Aplikace je čistě konzolová. Funguje buď interaktivně, nebo
dávkově. Spustí-li se aplikace bez parametru, bude čekat na vstup
uživatele. Každý příkaz oddělí uživatel stisknutím klávesy ENTER.

Výstup pak vypadá nějak takto:
\begin{verbatim}
[1]> (car (list (list 1 2 3) 4 5))
(1 2 3)
[2]> (set 'x1 2)
2
[3]> (set 'x2 (* x1 2))
4
[4]> (defun mul 2 (* #0 #1))
(* #0 #1)
[5]> (mul (list x1 x2))
8
\end{verbatim}

Program skončí se zadáním \texttt{(quit)}. V~tabulce \ref{tab:prik}
jsou všechny podporované příkazy. Vytvořit jde také komplexnější příkaz
jako je třeba definice faktoriálu nebo Fibonacciho posloupnosti:
\begin{verbatim}
(defun fact 1 (if (>= #0 2)
  (* #0 (fact (- #0 1))) (print 1)))

(defun fib 1  (if (>= #0 3) (+ (fib (- #0 1))
  (fib (- #0 2))) (print 1)))
\end{verbatim}

\begin{table}
\centering
\begin{tabular}{|r|l|}
\hline
quote & argument nebude vyhodnocen\\ \hline
set & uloží hodnotu proměnné\\ \hline
+ - * / & vypočte aritmetický výraz\\ \hline
= != \textless{=} \textgreater{=} \& | & vyhodnotí booleovský výraz\\ \hline
if & vyhodnotí podmínku\\ \hline
car & vrátí první prvek seznamu\\ \hline
cdr & vrátí všechny prvky seznamu bez prvního\\ \hline
list & vrátí seznam\\ \hline
defun & definuje funkci (defun \textbf{jméno} \textbf{počet-par} \textbf{tělo})\\ \hline
quit & ukončí program\\ \hline
help & vypíše informace o~autorovi a příkazech\\
\hline
\end{tabular}
\caption{Podporované příkazy interpretu}
\label{tab:prik}
\end{table}

Všechny operátory se zapisují prefixově. To znamená že prvním symbolem je operátor
a pak následují všechny následující, na které se postupně aplikuje operace.
Odlišně jsem pojal předávání argumentů. Jedná se totiž o~konstantní předávání
hodnotou. Při definici funkce zadá uživatel počet parametrů a do kódu
funkce zapisuje placeholder podle pořadí v~seznamu, který předá při volání.

\section{Závěr}
Při vypracování jsem vycházel z~původní semestrální práce pro předmět KIV/PC.
Jedná se tedy o~druhou generaci této aplikace ve které jsem vylepšil nedokonalosti
té předchozí. Vylepšil jsem jak lexikální část, datové struktury, čtení
z~různorodých vstupů a hlavně syntaktickou analýzu. Přepsal jsem prakticky
celou aplikaci.

Mezi přidanou funkcionalitou je především podpora funkcí, vyhodnocování podmínek,
mnoho oprav v~syntaxi a způsob ošetřování výjimek s~výpisem příslušné chyby.

Pro ladění jsem využil debugger \textsf{gdb} a vyhodnotil případné problémy v~programu
\textsf{valgrind}. Dokumentace byla napsána v~typografickém systému
\LaTeX{}. Protože jsem dělal minimálně na dvou počítačích, verzoval jsem
na GitHubu na adrese
\url{https://github.com/qwertzdenek/AritmeticSolver}. Program bude dále
veřejně dostupný pod licencí GNU GPL verze 3.

\end{document}
