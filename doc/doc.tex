\documentclass{article}

\usepackage{czech}
\usepackage{graphicx}
\usepackage[utf8]{inputenc}

\begin{document}
\begin{titlepage}
\includegraphics[width=30ex]{fav_cmyk}
\vfill
\begin{center}
{\huge Programování v jazyce C}\\[2ex]
{\Large Interpret podmnožiny jazyka LISP}
\end{center}
\vfill
\begin{tabbing}
Vypracoval: \hspace{1ex}\=Zdeněk Janeček\kill
Vypracoval: \>Zdeněk \textsc{Janeček}\\
Datum:\> \today
\end{tabbing}
\end{titlepage}

\section{Problematika}
Každý programovací popisujeme gramatikou jako každý jíný jazyk.
Prefixový zápis celou věc zjednodušuje, narozdíl od běžného přístupu.
LISP nám už v historii ukázal že lze programovat lehce jinak.  Totiž,
že vše je seznamem.

\begin{figure}
START $\longrightarrow$ VÝRAZ $|$ KOMP\\
VYRAZ  $\longrightarrow$ $\pm$ KOMP RETEZ\\
KOMP  $\longrightarrow$ č $|$ IDENT $|$ (VYRAZ) $|$ (KOMP)\\
RETEZ  $\longrightarrow$ KOMP RETEZ $|$ KOMP\\
IDENT  $\longrightarrow$ QUOTE $|$ quit $|$ set $|$ car $|$ *\\
QUOTE  $\longrightarrow$ č $|$ ř $|$ op $|$ z
\end{figure}

\end{document}
